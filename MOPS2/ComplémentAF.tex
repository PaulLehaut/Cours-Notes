\documentclass{article}

\title{Complément d'Analyse Fonctionnelle}
\author{Paul Lehaut}
\usepackage{mesraccourcis}

\begin{document}
\maketitle
\newpage
\tableofcontents
\newpage
\section{Convergence Faible dans les Espaces de Hilbert}
Dans tout ce chapitre $\mc H$ désigne un espace de Hilbert.
\bigskip 

\Def \ Convergence faible dans un espace de Hilbert
\sskip
Soient $(u_n)\in\mc H^\N$ et $u\in\mc H$, on dit que:
\begin{itemize}
    \item $(u_n)$ converge fortement vers $u$ dans $\mc H$, ce qu'on note $u_n\to u$ si et seulement si $\lim||u_n-u||_\mc H =0$
    \item $(u_n)$ converge faiblement vers $u$ dans $\mc H$, ce qu'on note $u_n\rightharpoonup u$ si et seulement si: $$\forall v\in\mc H, \ \lim (v,u_n)_\mc H=(v,u)_\mc H.$$
\end{itemize}
\bigskip
\subsection{Notions de Topologie Générales}
On définit tout d'abord le concept même d'une topologie et des concepts fonadamentaux:
\smallskip 

\Def 
\sskip 
Soit $X$ un ensemble, un ensemble $\mc O\subset \mc P(X)$ est appelé une topologie si:
\begin{itemize}
    \item $\mc O$ contient $X$ et $\emptyset$
    \item $\mc O$ est stable par union queclconque
    \item $\mc O$ est stable par intersection finie.
\end{itemize}
\smallskip
Un élément de $\mc O$ est alors appelé un ouvert, son complémentaire est un fermé.
\sskip 
Pour deux topologie $\mc O_1, \ \mc O_2$, on dit que $\mc O_1$ est la plus fine si $\mc O_2\subset\mc O_1$.
\sskip 
Soit $x\in X$, un ensemble $V_x$ est appelé un voisinage de $x$ s'il contient un ouvert contenant $x$.
\sskip 
Une topologie est dite séparée si elle vérifie l'axiome de Hausdorff: pour tout élément distinct $x$ et $y$ il existe deux voisinages de $x$ et $y$ disjoints.
\bskip 
Différentes topologies permettent de définir différentes notions de continuité et de convergence.
\smallskip 

\Def \ Convergence dans un espace topologique
\sskip 
On dit qu'une suite $(x_n)$ converge vers $x$ pour la topologie $\mc O$ si et seulement si: pour tout voisinage de $x$, il existe un certain range de $(x_n)$ à partir duquel $(x_n)$ est incluse dans ce voisinage.
\bskip 
Il est intéressant de noter que, dans un espace topologique non séparé, une suite peut avoir plusieurs limites.
\bigskip 

\Def \ Continuité 
\sskip 
Soient $(X,\mc O_X)$ et $(Y, \mc O_Y)$ deux espaces topologiques, la fonction $f:X\to Y$ est dite continue en $x\in X$ si, pour tout voisinage $V_{f(x)}$ de $f(x)$, il existe $U_x$ un voisinage de $x$ tel que:
$$f(U_x)\subset V_{f(x)}$$
on dit que $f$ est continue si elle est continue en tout point $x$ de $X$.
\bigskip 

\Def \ Espace métrique et topologie métrique 
\sskip 
$X$ est un espace métrique si on peut le munir d'une distance $d:X\times X\to \R_+$ symétrique telle que: $d(x,y)\le d(x,z) + d(z,y)$ et $d(x,y)=0\Leftrightarrow x=y$.
\sskip 
On peut alors définir la topologie métrique qui est la topologie séparée définie par:
$$\mc O_d = \{\Omega \subset X \ |\ \forall x\in \Omega, \ \exists \epsilon >0:\ \mc B(x,\epsilon)\subset \Omega \}.$$

\Def \ Topologie forte d'un \ev \ normé 
\sskip 
Soit $X$ un \ev \ normé et $d$ la distance associée à la norme, la topologie métrique de $X$ est appelée la topologie forte de $X$.
\bigskip 

\Def \ Ensemble compact selon Borel-Lebesgue 
\sskip 
Soient $(X,\mc O)$ un espace topologique séparé et $K\subset X$, on dit que $K$ est un compact s'il vérifie la propriété de Borel-Lebesgue c'est-à-dire que de tout recouvrement d'ouvert de $K$ on peut extraire un sous-recouvrement fini.
\bigskip 

\Theo \ Ensemble compact selon Bolzano-Weiestrass 
\sskip 
On considère le cas particulier où $X$ est un espace métrique, alors $K\subset X$ est compact si et seulement si il vérifie la propriété de Bolzano-Weiestrass c'est-à-dire que de toute suite d'élément de $K$ on peut extraire une sous-suite convergente dans $K$.
\bigskip 
\subsection{Convergence Faible et Topologie Faible}
Soit $\mc Z$ un ensemble de cardinal fini d'éléments d'un espace de Hilbert $\mc H$, pour $r>0$ on pose:
$$V_{u,r,\mc Z}=\{v\in \mc H \ | \ \forall w\in \mc Z, \ |(w,v-u)_\mc H|<r\}$$
et:
$$\mc O_{\text{faible}}=\{\Omega\subset\mc H \ |\ \forall u\in \Omega, \ \exists r>0 \ \ \text{et}\ \ \exists \mc Z\subset \mc H \ \ \text{fini:}\ V_{u,r\mc Z}\subset \Omega\}.$$
Alors on a les résultats suivants:
\begin{itemize}
    \item $\mc O_{\text{faible}}$ est une topologie sur $\mc H$, $V_{u,r,\mc Z}$ est un voisinage ouvert de $u$ pour la topologie faible
    \item $\mc O_{\text{faible}}$ est séparée et moins fine que la topologie forte, c'est en fait la topologie la moins fine assurant la continuité de toutes les formes linéaires (fortement) continues sur $\mc H$
    \item Une suite de $\mc H$ converge au sens de la topologie faible si et seulement si elle converge faiblement
    \item Si $\mc H$ est de dimension finie, alors $\mc O_{\text{faible}}=\mc O_{\text{forte}}$.
\end{itemize}
Intuitivement, l'ensemble $\mc Z$ correspond à un ensemble d'observateurs par rapport auxquels on considère les distances. Le point $u$ correspond alors au centre du voisinage, prenons par exemple $\mc H=\R^3$, $u=0$, $r=1$ et $\mc Z=\{(1,0,0)\}$, alors $V_{u,r,\mc Z}$ est la plaque infinie selon $y,z$ avec $x\in(-1,1)$.
\bigskip 
\subsection{Compacité Faible des Suites Bornées}
On commence par rappeler un théorème fondamental:
\smallskip 

\Theo \ Existence de bases hilbertiennes dans un espace de Hilbert séparable 
\sskip 
Soit $\mc H$ un espace de Hilbert séparable, alors il existe une famille $(e_k)$ d'éléments de $\mc H$ telle que:
\begin{itemize}
    \item Pour tout $(k,l)\in\N^2, \ (e_k,e_l)=\delta_{k,l}$
    \item $\overline{Vect((e_k)_k)}=\mc H$.
\end{itemize}
Une telle famille est appelée une base hilbertienne de $\mc H$. On a alors, pour tout $u\in\mc H$:
\begin{itemize}
    \item $u=\sum_{n\in\N}(e_n,u)e_n$
    \item $||u||^2_\mc H = \sum_{n\in\N}|(e_n,u)_\mc H|^2$, il s'agit de la formule de Perseval.
\end{itemize}
\bigskip 

\Theo \ Compacité faible des suites bornées 
\sskip 
De toute suite bornée d'éléments de $\mc H$ on peut extraire une sous-suite $(u_{n_k})$ qui converge faiblement vers $u\in \mc H$.
\smallskip 

\underline{Démonstration:} 
\sskip
Soit $(u_n)$ une suite bornée de $\mc H$, on note $C$ sa borne supérieure. On considère $(e_n)$ une base hilbertienne de $\mc H$ (si $\mc H$ n'est pas séparable on peut simplement étudier $\overline{Vect((u_n)_n)}$ qui, muni du produit scalaire de $\mc H$, est un espace de Hilbert séparable).
\sskip 
On a, pour tout $n\in \N$:
$$u_n=\sum_{k\in \N}(e_k,u_n)_\mc H e_k \ \ \text{avec} \ \  ||u_n||_\mc H^2=\sum_{k\in\N}|(e_k,u_n)|^2\le C^2.$$
\sskip 
Pour tout entier naturel $k$, la suite $((e_k,u_n))_n$ est donc une suite bornée de $\R$ ou $\C$. Par procédé diagonal, on peut donc extraire de $(u_n)$ une sous-suite $(u_{n_j})$ telle que, pour tout $k$, $((e_k,u_{n_j})_j)$ converge vers $w_k$.
\sskip 
Il vient alors:
$$\sum_{k\in\N}|w_k|^2\le C^2.$$
On peut donc définir $w=\sum_{k\in\N}w_ke_k$.
\sskip 
Soit maintenant $v\in \mc H$, alors:
$$(v,u_{n_j})_\mc H =\sum_{k\in\N}(e_j,u_{n_j})_\mc H(v,e_k)_\mc H\longrightarrow_{j\to\pinf} \sum_{k\in\N}w_k(v,e_k)=(v,w)_\mc H$$
donc $(u_{n_j})_j$ converge faiblement vers $w$.
\bigskip 

\subsection{Quelques Résultats Essentiels}
On introduit les résultats suivants:
\smallskip 

\Theo 
\begin{itemize}
    \item Toute suite faiblement convergente dans $\mc H$ est bornée.
    \item Si $(u_n)$ converge faiblement vers $u$, alors: $||u||_\mc H\le\lim\inf_n||u_n||_\mc H$.
    \item Si $(u_n)$ converge faiblement vers $u$ et si $(v_n)$ converge fortement vers $v$, alors: $\lim(v_n,u_n)_\mc H=(v,u)_\mc H$.
\end{itemize}
\bigskip 

\Theo \ Mazur 
\sskip 
Soit $K$ un sous-ensemble de $\mc H$ convexe et fermé pour la topologie forte, alors $K$ est fermé pour la topologie faible. En particulier si une suite d'éléments de $K$ converge faiblement dans $\mc H$, alors sa limite est dans $K$.
\bigskip 

\Theo 
\sskip 
Soit $J:\mc H\to \R$ un fonctionnelle continue et convexe telle que: $\lim_{||u||_\mc H\to \pinf}J(u)=\pinf$. Soit $K$ un sous-ensemble fermé convexe et non-vide de $\mc H$, alors $J$ admet un minimiseur global sur $K$. De plus, tout minimiseur local de $J$ sur $K$ est global.
\sskip 
L'ensemble des minimiseurs de $J$ sur $K$ est un sous-ensemble convexe et fermé de $K$.
\bigskip 

\subsection{Convergence Faible dans $L^2$ et $H^1$}
Tout d'abord, il est claire que si $(u_n)\in (L^2)^\N$ converge failbement vers $u$ dans $L^2$ vers $u$, alors la convergence a également lieux dans $\mc D'$. 
\sskip 
En effet, si $\Phi\in\mc D$, alors $\Phi\in L^2$ et il vient:
$$<u_n,\Phi>=(u_n,\Phi)_{L^2}\to(u,\Phi)_{L^2}=<u,\Phi>.$$
On introduit alors le lemme suivant:

\underline{Lemme:}
\sskip 
Soit $\Omega$ un ouvert de $\R^d$ et soit $(u_n)\in H^1(\Omega)^\N$ convergeant faiblement dans $H^1(\Omega)$ vers $u$. Alors $(u_n)$ converge faiblement vers $u$ dans $L^2(\Omega)$.
\smallskip 

\underline{Démonstration:}
Soient $\Phi\in L^2(\Omega), \ w_\Phi\in H^1(\Omega)$ le représentant de Riesz de la forme linéaire continue $L_\Phi$ sur $H^1$ définie par:
$$\forall v\in H^1(\Omega), \ L_\Phi(v)=(v,\Phi)_{L^2(\Omega)}$$
on a donc: $\forall v\in H^1(\Omega), \ (w_\Phi,v)_{H^1}=L_\Phi(v)=(v,\Phi)_{L^2}$,
\newline 
et donc: $(u_n,\Phi)_{L^2} = (w_\Phi,u_n)_{H^1} \to (w_\Phi,u)_{H^1}=(u,\Phi)_{L^2}$.
\bigskip 

\section{Espace de Sobolev}
On utilisera également dans ce chapitre les espaces de Hölder définis par, pour $\Omega\subset \R^d$ un ouvert:
$$\forall 0 <\alpha \le 1, \ \mc C^{0,\alpha}(\bar \Omega) = \{u\in\mc C^0(\bar \Omega)| ||u||_{\mc C^{0,\alpha}} = ||u||_{\mc C^{0}} + \sup_{x\neq y}\frac{||u(x)- u(y)||}{|x-y|^\alpha}<\pinf\}$$
$$\text{avec} \ \ ||u||_{\mc C^{0}} = \sup_{x\in\bar\Omega}||u(x)||$$
et:
$$\forall 0 <\alpha \le 1, \ m\in \N, \ \mc C^{m,\alpha}(\bar \Omega) = \{u\in\mc C^m(\bar \Omega)| ||u||_{\mc C^{m,\alpha}} = \max_{|\beta|\le m}||\partial^\beta u||_{\mc C^{0}} + \max_{|\beta|=m}||\partial^\beta u ||_{\mc C^{0,\alpha}}<\pinf\}.$$
On notera que l'\ev \ des fonctions de $\Omega\longrightarrow \R^d$ de classe $\mc C^{\infty}$ à support compact dans $\Omega$ ou bien $\mc C_c^{\infty}(\Omega)$ ou bien $\mc D(\Omega)$ selon qu'on l'interprète comme l'ensemble des fonctions test ou non.
\bigskip

\subsection{Espace de Sobolev d'Ordre Entier}
Il s'agit d'une généralisation naturelle des espaces de Sobolev $H^k$ vus en première année.
\bigskip 

\Theo 
\sskip 
Soient $k\in\N^*$ et $1\le p\le \pinf$. On note $W^{k,p}(\Omega)$ l'espace de Sobolev:
$$W{k,p}(\Omega)=\{u\in L^p(\Omega)|\ \forall  \alpha \in \N^d, |\alpha|\le k, \ \partial^\alpha u\in L^p(\Omega)\}$$
on le muni de la norme:
$$||u||_{W^{k,p}} = (\sum_{\alpha\in\N^d,\ |\alpha|\le k} = ||\partial^\alpha u||^p_{L^p})^{1/p} \ \ \text{si} \ \ p<+\infty, \ \ \ ||u||_{W^{k,\infty}}=\max_{\alpha\in\N^d, \ |\alpha|\le k}||\partial^\alpha u||_{L^\infty} \ \ \text{sinon} $$
alors il s'agit d'un espace de Banach.
\bigskip 

\Def 
\sskip 
Pour $p=2$, on note $H^k(\Omega)=W^{k,2}(\Omega)$, on munit cette espace du produit scalaire:
$$(u,v)_{H^k} = \sum_{|\alpha|\le k}(\partial^\alpha u, \partial^\alpha v)_{L^2}$$
on remarque en particulier que: $H^1(\Omega)=\{u\in L^2(\Omega)|\ \nabla u\in L^2(\Omega)\}$.
\bigskip 

\Prop 
\sskip 
Pour $u\in H^1$, alors: $||u||_{H^1}^2 = ||u||_{L^2}^2 + ||\nabla u||_{L^2}^2$.
\bigskip 

\Theo 
\sskip 
Pour tout $k \in \N^*$, $H^k(\Omega)$ est un espace de Hilbert séparable.
\smallskip

\Dem \ Pour $H^1$
\sskip 
Soit $(u_n)$ une suite de Cauchy dans $H^1$, alors, comme:
$$||u_{n+p} - u_{n}||_{L^2}\le||u_{n+p} - u_{n}||_{H^1} \ \ \text{et} \ \ ||\nabla u_{n+p} - \nabla u_{n}||_{L^2}\le ||u_{n+p} - u_{n}||_{H^1}$$
on en déduit que les suites $(u_n)$ et ($\partial^j u_n)$ sont des suites de Cauchy dans $L^2$, or $L^2$ est Complet, il existe donc $u$ et $v_j$ telles que:
$$u_{n}\longrightarrow u \ \ \text{et} \ \ \frac{\partial u_{n}}{\partial x_j}\longrightarrow v_j. $$
Or, la convergence dans $L^2$ implique la convergence dans $\mc D'(\Omega)$ et $u_n\longrightarrow u$ dans $\mc D'(\Omega)$ implique:
$$\frac{\partial u_n}{\partial x_j}\longrightarrow \frac{\partial u}{\partial x_j}$$
donc, comme $D'(\Omega)$ est séparé, alors la limite est imite et $v_j = \frac{\partial u}{\partial x_j}$, donc $\nabla u = (\ind v n)^T\in L^2(\Omega)$.
On en déduit donc que $u\in H^1(\Omega)$ et: 
$$||u_n-u||_{H^1} = ||u_n-u||_{L^2} + ||\nabla u_n - \nabla u||_{L^2} \ \ \text{donc} \ \ u_n\longrightarrow u \ \ \text{dans} \ \ H^1(\Omega).$$
\bigskip 

\Def 
\sskip 
On note $W_0^{k,p}(\Omega)=\overline{\mc C_c^\infty(\Omega)}^{||\cdot||_{W^{k,p}(\Omega)}}$. Il s'agit d'un sous-\ev \ fermé de l'espace de Banach $W^{k,p}(\Omega)$, c'est donc également un espace de Banach (pour la norme $||\cdot||_{W^{k,p}(\Omega)}$).
\bigskip 

\subsection{IPP en Dimension d et Formule de Stokes}
Soit $\Omega$ un ouvert de $\R^d$.
\smallskip 

\Def 
\sskip 
$\Omega$ est lipschitzien si, pour tout $x$ de sa frontière, il existe un voisinage $V_x$ de $x$ et un homémorphisme 
\newline
$\Phi_x : (-1,1)^d\to V_x$ tel que $\Phi_x$ et $\Phi^{-1}_x$ soient lipschitziennes avec $\Phi_x(0)=x$ et $\Phi_x((-1,1)^{d-1}\times(-1,0)) =V_x\cap \Omega$.
\sskip 
On dit que $\Omega$ est uniformément lipschitzien si les constantes de lipschitz de $\Phi_x$ et $\Phi^{-1}_x$ peuvent être choisies uniformément bornées.
\sskip 
On dit que $\Omega$ est de classe $\mc C^k$ si $\Phi_x$ et son inverse peuvent être choisies de classe $\mc C^k$.
\sskip 
On dit que $\Omega$ est régulier si il est de classe $\mc C^k$ pour tout $k\in \N$.
\bigskip 

\Theo \ Stokes 
\sskip 
Si $\Omega$ est lipschitzien, on peut définir $\sigma$ une mesure de surface sur sa frontière et une normale sortante $n(x)$ presque partout (pour la mesure $\sigma$) telles que:
$$\forall X\in\mc C^1(\bar\Omega, \R^d), \ \int_\Omega \text{div} X = \int_{\partial \Omega}X\cdot n d\sigma.$$
\subsection{Théorèmes de Traces}
La théorie des distributions est idéale pour traiter des problèmes linéaires posés dans tout l'espace, elle se révèle par contre inadaptée pour les études aux limites ou pour les problèmes non linéaires.
L'objectif des espaces de Sobolev est de lever ces contraintes.
\bskip 
Une fonction de $L^2(\Omega)$ n'a pas forcément de valeur bien définie sur le bord de $\Omega$. 
\sskip 
En dimension 1, une fonction de $H^1((a,b))$ admet un unique représentant continue dans $\mc C^0([a,b])$. On peut donc donner un sens à ses valeurs aux bords.
\sskip 
Dans des dimensions supérieures néanmoins, les fonctions de $H^1(\Omega)$ n'ont en général pas de représentant continu. Si $\Omega$ est uniformément lipschitzien, on peut toutefois donner un sens aux valeurs aux bords de ces fonctions.
\smallskip

\Theo \ De la trace 
\sskip 
On suppose donc $\Omega$ uniformément lipschitzien dans $\R^d$, alors l'application linéaire de $H^1(\Omega)\cap\mc C^p_c(\bar\Omega)$ dans $L^2(\partial\Omega)$ définie par
$$u\mapsto u_{|\partial\Omega}$$
admet un unique prolongement par continuité $\gamma_0:H^1(\Omega)\to L^2(\partial\Omega)$, son noyau est $H_0^1(\Omega)$.
\sskip 
Enfin, on a: 
$$\forall u,v\in H^1(\Omega), \ \int_\Omega \frac{\partial u}{\partial x_i}v = \int_{\partial\Omega}\gamma_0(u)\gamma_0(v)(n\cdot e_i)d - \frac{\partial v}{\partial x_i}u.$$

\Theo \ Dérivé normale sortante 
\sskip 
Sous la même hypothèse sur $\Omega$, l'application de $H^2(\Omega)\cap\mc C^1_c(\bar\Omega)$ dans $L^2(\partial\Omega)$ définie par: 
$$u\mapsto \frac{\partial u}{\partial x} = \nabla u \cdot x$$
admet également un unique prolongement continu : $\gamma_n$ de $H^2(\Omega)$ dans $L^2(\partial\Omega)$.
\sskip 
De plus:
$$\forall u\in H^2(\Omega), \ v\in H^1(\Omega), \ \int_\Omega (-\Delta u)v = -\int_{\partial \Omega}\gamma_n(u)\gamma_0(v)+\int_\Omega\nabla u\cdot\nabla v.$$

\Def 
\sskip 
On suppose que $\Omega$ est un ouvert quelconque de $\R^d$, alors: $H_{div}(\Omega)=\{v\in (L^2(\Omega))^d|\text{ div}v\in L^2(\Omega)\}$, on munit cette espace du produit scalaire:
$$(v,w)_{H_{div}(\Omega)}=\int_\Omega v\cdot w + \int_\Omega \text{div} v\cdot \text{div}w$$
alors $H_{div}(\Omega)$ est un espace de Hilbert.
\bigskip 

\Theo \ Flux normal sortant 
\sskip 
L'application linéaire de $H_{div}(\Omega)\cap \mc C^1_c(\Omega)$ dans $H^{1/2}(\partial \Omega)^*$ (dual de $H^{1/2}(\Omega)$) définie par:
$$v\mapsto v\cdot n$$
se prolonge en une unique application linéaire continue $\gamma_\phi :H_{\text{div}}(\Omega)\to H^{-1/2}(\partial\Omega)$.
\sskip 
On a alors la formule:
$$\forall v\in H_{div}(\Omega), \ \forall u\in H^1(\Omega), \ \int_\Omega \text{div}v\ u = <\gamma_\phi,\gamma_0(u)>_{H^{-1/2}(\partial\Omega)}-\int_\gamma v\cdot\nabla u.$$
\subsection{Injections Continues de Sobolev}
On rappelle que pour $V$ et $W$ des espaces-vectoriels \ normés, la notation $V\hookrightarrow W$ s'injecte de façon naturelle dans $W$ et qu'il existe une constante $C\ge 0$ telle que: $\forall u \in V$, $||u||_W\le C||u||_V$.
\bigskip 

\Theo \ Sobolev - Gagliadov - Niremberg 
\sskip 
Soit un entier $d$ et $1\le p< d$, on a $W^{1,p}(\R^d)\hookrightarrow L^{\frac{dp}{d-p}}(\R^d)$.
\bigskip 

\Theo 
\sskip 
$$W^{1,d}(\R^d)\hookrightarrow L^q(\R^d), \ 2\le q<\pinf.$$

\Theo \ Money 
\sskip 
Soit $\alpha = 1-\frac{d}{p}\in (0,1)$ pour $d<p\le \pinf$, alors:
$$W^{1,p}(\R^d)\hookrightarrow L^\infty (\R^d)\cap C^{0,\alpha}(\R^d) \ \ \text{avec} \ \ C^{0,\alpha}(\R^d)=\{u\in\mc C^0(\R^d)|\ \sup_{x,y\in\R^d}\frac{|u(x)-u(y)|}{|x-y|^\alpha}<\pinf\}.$$

\Theo \ Interpolation 
\sskip 
On suppose que $\Omega$ est un ouvert de $\R^d$, soient $1\le p\le q\le\pinf$, si $f\in L^p(\Omega)\cap L^q(\Omega)$, alors, pour tout $p\le r\le q$, $f\in L^r(\Omega)$ et:
$$||f||_{L^r}\le||f||_{L^p}^\alpha||f||_{L^q}^{1-\alpha}, \ \ \text{avec}\ \ \frac{1}{r} = \frac{\alpha}{p}+\frac{1-\alpha}{q}.$$
\end{document}