\documentclass{article}

\title{Statistiques et Analyse de Données}
\author{Paul Lehaut}
\usepackage{mesraccourcis}

\begin{document}
\maketitle
\newpage
\tableofcontents
\newpage
\section{Introduction}
La théorie des jeux est une branche des mathématiques qui étudie les interactions stratégiques. Elle s'applique dans de nombreux domaines comme l'économie, les sciences politiques, la biologie...
\sskip 
Le rôle de la théorie des jeux est à la fois de modéliser et d'analyser les situations. Idéalement, les compertements purement individuels permettraient d'atteindre des objectifs collectifs.
\bigskip 
\section{Jeux Sous Forme Normale: Stratégies Pures}
La forme normale est une manière standard de décrire formellement un jeu. On spécifie pour chaque joueur son ensemble de stratégies et sa fonction de paiement.
\sskip 
Dans ce chapitre (et le suivant), on supposera que les joueurs connaissent la description exacte du jeu.
\sskip 
On définit deux types de stratégies:
\begin{itemize}
    \item stratégie pure: spécifie, pour chaque situation, une décision bien déterminée
    \item stratégie mixte: elle intègre des choix aléatoires
\end{itemize}
dans ce chapitre, on se restreindra à des stratégies pures (désignées dès lors par stratégie).
\bigskip 
\subsection{Description d'un Jeu Sous Forme Normale}
Un jeu sous forme normale est défini pour un nombre fini $N$ de joueurs par:

\Def \ Jeu sous forme normale 
\sskip 
Un ensemble de stratégies $S_i$ pour chaque joueur $i$ et une fonction de paiement $g_i:\prod_{j=1}^N S_j\to\R$.
\sskip 
L'interprétation est la suivante: chaque joueur joue une fois, indépendamment des autres joueurs, et reçoit ensuite un paiement.
\bskip 
Il est intuitif de considérer que chaque joueur cherche à maximiser son profit. On étudie donc la question suivante: 

Si les joueurs sont rationnels, quelles stratégies vont-ils le plus probablement choisir ?
\bskip 
Dans ce chapitre, on se limite à des jeux statiques (les joueurs jouent une unique fois) et à information complète (les joueurs connaissent les paiements et stratégies des autres joueurs). On définit les classes de jeux particulièrement importantes suivantes:
\begin{itemize}
    \item Jeux à intérêts communs: tous les joueurs ont les mêmes intérêts ou préférences.
    \item Jeux à somme nulle: jeux à deux joueurs dont les intérêts sont antagonistes $ (g_1 = -g_2)$.
    \item Jeux de la bataille des sexes: ces jeux font intervenir une part de coordination et de conflit entre les agents.
    \item Jeux de la fureur de vivre: deux adolescents en voiture foncent l'un vers l'autre, aucun d'entre eux ne veut sortir de la route, si les deux sortes le résultat est neutre.
    \item Le dilemme du prisonnier: ce jeux fait ressortir une tension entre l'intérêt collectif et individuel, de nombreuses situations possèdent une structure similaire à celle-ci.
    \item Compétition en quantités dite de Cournot: deux entreprises produisent en quantité $q_i$ et à un coût $c_i(q_i)$, des biens identiques, le prix résultant de la loi de l'offre et de la demande est $p(q_1 + q_2)$, la situation se formalise alors par:
$$S_i = \R_+ \ \ \text{et} \ \ g_i(q_i, q_j) = q_i p(q_i + q_j) - c_i(q_i).$$
    \item Compétition en prix dite de Bertrand: il s'agit d'un modèle de compétition par les prix, chaque entreprise décide d'un prix de vente du bien et les consommateurs d'une quantité à acheter à la firme au prix le plus bas.
\end{itemize}
\bigskip 
\subsection{Stratégies Dominantes et Dominées}
Dans toute la suite on utilisera les notations suivantes: 
$$S = \prod_{j=1}^NS_j, \ \ S_{-i} = \prod_{j=1, \ j\ne i}^N S_j \ \ \text{et} g = (g_i).$$
on considèrera par ailleurs un jeu sous forme normale $\Gamma(N,S,g)$.
\bigskip 

\Def \ Stratégie strictement dominée 
\sskip 
Il s'agit d'une stratégie $s_i\in S_i$ telle qu'il existe $t_i\in S_i$ qui domine strictement $s_i$, c'est-à-dire:
$$\forall s_{-i}\in S_{-i}, \ g_i(t_i,s_{-i})>g_i(s_i,s_{-i}).$$
Un joueur 'rationnel' ne devrait jamais joué une stratégie strictement dominée. 
\bskip 
Une stratégie strictement dominante est une stratégie qui domine strictement toutes les autres, une telle stratégie, si elle existe, est évidemment unique.
\bigskip

\Def \ Equilibre en stratégies strictement dominantes
\sskip 
Il s'agit d'un profil de stratégies $s=(\ind s N)$ tel que, pour tout $i$, $s_i$ soit strictement dominante.
\bskip 
Si une telle stratégie existe, alors on peut considérer qu'il s'agit de la seule issue rationnelle du jeu.
On peut, par ailleurs, considérer une jeu $\Gamma(N,S^*,g)$ dans lequel on ne conserve que les stratégies strictement dominantes (si elles existent). Ainsi, si, pour tout $i$, $S_i^* = \{s_i^*\}$, alors le profil de stratégies $(s_i^*)$ ne dépend pas de l'ordre d'élimination des stratégies, on dit que le jeu est résoluble par élimination itérée des stratégies strictement dominées.
\sskip 
Ainsi, lorsqu'un jeu est résoluble par élimination itérée des stratégies strictement dominées, on peut lui associer une prédiction unique des stratégies suivies par les joueurs: le profil $(s_i^*)$.
\bskip 
Toutefois, il n'est pas rare de se retrouver dans une situation où il n'existe pas (ou plus) de stratégie absolument dominée, on introduit donc le concept de stratégie dominée:
\smallskip

\Def \ Stratégie dominée 
\sskip 
Il s'agit simplement d'une stratégie pour $s_i$ telle qu'il existe $t_i$ telle que:
$$\forall s_{-i}\in S_{-i}, \ g(s_i,s_{-i})\le g(t_i,s_{-i}).$$
Un stratégie dominante représente un choix raisonnable (on ne peut pas le regretter). On peut alors définir comme précédemment les profils de stratégies dominantes, lorsqu'un tel profil existe, il semble raisonnable que les joueurs le jouent.
\sskip 
On peut donc, comme précédemment, penser à supprimer les stratégies dominées de l'ensemble des stratégies, néanmoins, cette fois, l'ordre des suppressions influence le résultat.
\bigskip 
\subsection{Equilibre de Nash}
Dans le cas général, un jeu n'a que peu de chance d'être résoluble par élimination itérée des stratégies strictement dominées. Pour étudier ces jeux, on introduit le concept d'équilibre de Nash. Cette équilibre peut être interprété comme une convention sociale stable, on introduit les notations suivantes: si $s\in S$ et $i\in [|1,N|]$, on note $s_{-i}$ le profil de stratégies où on a retiré la stratégie du joueur $i$. On peut alors définir l'équilibre de Nash:
\smallskip 

\Def \ Equilibre de Nash 
\sskip 
Il s'agit d'un profil de stratégies $s$ tel que:
$$\forall i,\ \forall t_i\in S_i, \ g_i(s_i,s_{-i})\ge g_i(t_i,s_{-i}).$$
On notera $NE(\Gamma)$ l'ensemble des équilibres de Nash de $\Gamma$ et $E(\Gamma)$ l'ensemble des paiements pour ces équilibres.
\bskip 
On peut reformuler la définition d'équilibre de Nash à l'aide de la notion de meilleure réponse:

\Def \ Meilleure réponse 
\sskip 
$s_i$ est la meilleure réponse à $s_{-i}$ si, pour tout $t_i$, $g_i(s_i,s_{-i})\ge g_i(t_i,s_{-i})$.
\sskip 
L'ensemble des meilleures réponses à $s_{-i}$ est noté $BR_i(s_{-i})$, on peut alors introduire la reformulation suivante:

\Prop 
\sskip 
Un profil de stratégie $s\in S$ est un équilibre de Nash si et seulement si pour tout $i$, $s_i\in BR_i(s_{-i})$.
\bskip
On peut alors se demander quel est le lien entre l'équilibre de Nash et les notions de solutions rationnelles vues précédemment ? 
\bigskip 

\Prop 
\sskip 
\begin{itemize}
    \item Un équilibre en stratégies strictement dominantes est l'unique équilibre de Nash, le profil de stratégies obtenu est alors celui de l'équilibre de Nash.
    \item Soit $(\Gamma^k)_k$ la suite de jeux obtenus par élimination successive des stratégies strictement dominées, alors $NE(\Gamma^k)=NE(\Gamma)$ pour tout $k$.
    \item Un équilibre en stratégies dominantes est un équilibre de Nash.
    \item Soit $(\Gamma^k)_k$ la suite de jeux obtenus par une élimination successive des stratégies dominées, alors $NE(\Gamma^{k+1})\subset NE(\Gamma^k)$ pour tout $k$.
\end{itemize}
\bigskip 
\section{Jeux Sous Forme Normale: Stratégies Mixtes}
Un équilibre de Nash ne peut être atteint que si chaque joueur n'a pas d'intérêt à dissimuler sa stratégie. La théorie des stratégies mixtes permet alors d'étudier les cas pour lesquels les joueurs on intérêt à bluffer, voire joue aléatoirement.
\bigskip 
\subsection{Définitions}
On supposera dans tout ce chapitre que chaque joueur ne possède qu'un nombre fini de stratégies.
\bigskip 

\Def \ Stratégies mixtes 
\sskip 
Il s'agit, pour le joueur $i$, d'une distribution de probabilités sur $S_i$. On note $\Sigma_i$ l'ensemble des stratégies mixtes, on appellera désormais les éléments de $S_i$ 'stratégies pures'.
\bskip 
Un stratégie pure $s_i$ peut-être vue comme une stratégie mixte avec probabilité $1$ de joueur $s_i$, on peut donc écrire $S_i\subset \Sigma_i$, comme $S_i$ est fini, alors on peut identifier:
$$\Sigma_i= \{p\in\R^{|S_i|} |\ \forall k, \ p_k\ge 0 \ \ \text{et } \sum_{k=1}^{|S_i|}p_k=1\}.$$

\Prop 
\sskip 
Cet ensemble est convexe.
\bskip 
Cette propriété joueura un rôle essentielle dans le théorème d'existence de l'équilibre de Nash en stratégies mixtes.
\bskip 
Si chaque joueur joue la stratégie $\sigma_i$, la probabilité que la stratégie pure $s=(\ind s N)$ soit le profil de stratégie effectivement joué est $\prod_{j=1}^N\sigma_j(s_j)$, le paiement espéré du joueur $i$ est alors:
$$\sum_{s\in S}(\prod_{j=1}^N\sigma_j(s_j))g_i(s).$$
Pour définir l'extension mixte d'un jeu, il reste finalement à dire que chaque joueur cherche désormais à maximiser son gain en moyenne (soit l'espérance de son paiement). La fonction $g_i:\Sigma \to \R$ ainsi définie est alors multilinéaire, en particulier, pour tout $i$ et $\sigma$, alors:
$$g_i(\sigma_i,\sigma_{-i}) = \sum_{s_i\in S_i}\sigma_i(s_i)g_i(s_i,\sigma_{-i}).$$

\Def \ Equilibre de Nash en stratégies mixtes 
\sskip 
Il s'agit d'un profil de stratégies mixtes $\sigma$ tel que, pour tout $i$:
$$\forall \tau_i\in\Sigma_i, \ g_i(\sigma_i,\sigma_{-i})\ge g_i(\tau_i,\sigma_{-i}).$$
Un tel équilibre peut s'interprété comme la satisfaction de chaque joueur vis-à-vis de sa probabilité, s'il connait les autres probabilités choisies. On notera $NE_{mix}(\Gamma)$ l'ensemble des équilibres de Nash en stratégies mixtes.
\bigskip 

\Prop 
\sskip 
$\sigma$ est un équilibre de Nash en stratégies mixtes si et seulement si:
$$\forall i, \ \forall s_i\in S_i, \ g(\sigma_i,\sigma_{-i})\ge g(s_i,\sigma_{-i}).$$
On en déduit alors que tout équilibre de Nash en stratégies pures en est également un en stratégies mixtes.
\bigskip 

\Theo \ Nash (considéré comme le théorème le plus important de la théorie des jeux)
\sskip 
Tout jeu fini (ie avec des ensembles de stratégies finis) admet un équilibre de Nash en stratégies mixtes.
\bskip 
On peut ensuite réintroduire les concepts de stratégie strictement dominée, dominée. On retrouve alors les propositions précendentes de relation entre les équilibres.
\bskip 
On peut, comme précédemment, éliminer une stratégie mixte qui est strictement dominée par une autre stratégie mixte. Néanmoins, cela ne possède pas de grand intérêt puisque les stratégies mixtes sont infinies. Par contre, il peut être intéressant de supprimer une stratégie pure et donc, par conséquent, toute les stratégies mixtes qui accordaient un poids à cette stratégie pure.
\bskip 
On présente désormais une méthode générale pour calculer tous les équilibres de Nash mixtes d'un jeu. On commence par réintroduire le concept de meilleure réponse:

\Def \ Meilleure réponse
\sskip 
$\sigma_i\in\Sigma_i$ est la meilleure réponse à $\sigma_{-i}\in \Sigma_{-i}$ si, pour tout $\tau_i\in\Sigma_i$, $g_i(\sigma_i,\sigma_{-i})\ge g_i(\tau_i,\sigma_{-i})$. 
\sskip 
L'ensemble des meilleures réponses à $\sigma_{-i}$ est $BR_i(\sigma_{-i})$.
\bskip 
On a alors la reformulation suivante: 

\Prop 
\sskip 
Un profil de stratégie $\sigma$ est un équilibre de Nash si et seulement si, pour tout $i$, $\sigma_i\in BR_i(\sigma_{-i})$.
\bskip 
On présente désormais un méthode de calcul des meilleures réponses (et donc des équilibres de Nash).
\bigskip 

\Def \ Support
\sskip 
Dans un jeu fini, le support de $\sigma_i$ est $supp(\sigma_i)=\{s_i\in S_i|\ \sigma_i(s_i)>0\}$.
\bigskip 

\Prop \ Principe d'indifférence faible
\sskip 
Soient $\sigma_{-i}\in\Sigma_{-i}$ et $\sigma_i\in BR_i(\sigma_{-i})$, alors:
$$\forall s_i,t_i\in supp(\sigma_i),\ g_i(s_i,\sigma_{-i}) = g_i(t_i,\sigma_{-i}).$$
Cette propriété est, par linéarité, équivalent à: $\forall s_i\in supp(\sigma_i)$, $g_i(s_i,\sigma_{-i}) = g_i(\sigma_i, \sigma_{-i})$.
\bskip 
On obtient alors une condition nécessaire pour être meilleure réponse. La proposition suivante donne une condition nécessaire et suffisante:

\Prop \ Principe d'indifférence fort 
\sskip 
Soit $\sigma_{-i}$, alors $\sigma_i\in BR_i(\sigma_{-i})$ si et seulement si:
$$\forall s_i,t_i\in supp(\sigma_i),\ g_i(s_i,\sigma_{-i}) = g_i(t_i,\sigma_{-i})\ \ \text{et } \forall s_i\notin supp(\sigma_i), \ g(\sigma_i, \sigma_{-i})\ge g(s_i,\sigma_{-i}).$$ 
Ainsi, $\sigma$ est un équilibre de Nash en stratégie mixte si et seulement si, pour tout $i$, $\sigma_i\in BR_i(\sigma_{-i})$.
\bskip 
Il suffit donc, pour trouver un équilibre de Nash, d'essayer tous les supports possibles, de résoudre les probabilités rendant chaque joueur indifférent sur son support, et de finalement vérifier que les stratégies hors du support ne donnent pas un paiement supérieur.
\bigskip 
\section{Jeux à Somme Nulle}
Dans un jeu à somme nulle, deux joueurs aux intérêts stratégiques opposés s'affrontent. Dans le cas où les deux joueurs possèdent un nombre finis de stratégie, on parle de jeu matriciel.
\bigskip 
\subsection{Valeur et Stratégie Optimale}
On se donnne dans toute cette partie un jeu à somme nulle $G=(I,J,g)$ et $w\in\bar \R$.
\bskip 
Nous introduisons tout d'abord le nouveau concept de valeur, pour ce faire on commence par introduire la notion de paiement garanti:

\Def 
\sskip 
Le joueur $1$ peut garantir le paiement $w$ s'il existe $i\in I$ tel que, pour tout $j\in J$, $g(i,j)\ge w$, l'égalité devient $g(i,j)\le w$ pour le joueur $2$.
\bigskip 

\Def \ Maxmim 
Le $maxmin$ de $G$ est le supremum des quantités garanties par le joueur $1$, on le note $\underline{v}$.
\sskip 
On définit le $minmax$ de façon analogue, on le note $\bar v$.
\bskip 
On peut alors introduire le concept clef de ce chapitre:

\Def \ Valeur
\sskip 
Le jeu $G$ a une valeur si $\underline v = \bar v$, cette quantité est notée $val(G)$.
\bskip 
Lorsque le jeu $G$ a une valeur, alors le joueur $1$ peut garantir au moins cette valeur, le joueur $2$ peut garantir au moins $-val(G)$. L'issu rationnelle est alors que le joueur $1$ gagne $val(G)$ (et le joueur $2$ $-val(G)$).
\bigskip 

\Prop 
\sskip 
Si les joueurs peuvent garantir la même quantité $w$, alors $G$ a une valeur égale à $w$.
\bigskip 

\Def 
\sskip 
Supposons que $G$ est une valeur, soit $\epsilon>0$, alors une stratégie du joueur $1$ ($2$) est dite $\epsilon$-optimale si elle garantit $val(G)-\epsilon$ ($val(G)+\epsilon$).
\bskip 
On note $I^*$ ($J^*$) les stratégies optimales du joueur $1$ ($2$).
\bigskip 
\subsection{Lien Entre Valeur, Stratégie Optimale et Equilibre de Nash}
Les concepts d'équilibre de Nash et de stratégie optimale sont essentiellement équivalents, comme le montre la proposition:

\Prop 
\sskip 
Soient $(i^*,j^*)\in I\times J$, alors on a l'équivalence: $G$ a une valeur et $(i^*,j^*)$ est un couple de stratégie optimale si et seulement si $(i^*,j^*)$ est un équilibre de Nash de $G$:
$$\forall (i,j)\in I\times J, \ g(i,j^*)\le g(i^*,j^*)\le g(i^*,j).$$ 
Lorsque ces assertions sont vérifiées, l'ensemble des paiements d'équilibre de Nash est $E=\{val(G), -val(G)\}$.
\bigskip 
\subsection{Jeux Matriciels et Théorème du Minmax}
Considérons un jeu $G=(g,I,J)$ avec $I$ et $J$ finis, on peut résoudre ce problème à l'aide de la méthode du chapitre $2$ (avec l'extension mixte de $G$). Nous allons néanmoins procéder différement.
\sskip 
On note pour ce faire $\Delta (C)$ l'ensemble des mesures de probabilité sur $C$.
\bigskip 

\Def 
\sskip 
L'extension mixte de $G$ est le jeu $\Gamma = (g^*, \Delta(I),\Delta(J))$ avec:
$$\gamma(\sigma,\tau)=\sum_{(i,j)\in I\times J}g(i,j)\sigma(i)\tau(j).$$
Dans la suie, on notera simplement $g^*$ $g$.
\bigskip 

\Prop \ (Analogue de tout Nash en stratégies pures est un Nash en stratégies mixtes)
\sskip 
Si une stratégie garantit $w$ dans $G$, alors cette stratégie garantit $w$ dans $\Gamma$. En particulier, si $G$ a une valeur, alors $\Gamma$ a la même valeur, et toute stratégie optimale de $G$ est une stratégie optimale de $\Gamma$.
\bigskip 

\Theo \ Von Neumann (corollaire du théorème de Nash)
\sskip 
Toute extension mixte d'un jeu matriciel admet une valeur, et chaque joueur a une stratégie optimale.
\bskip 
Ce théorème a été généralisé sous différentes formes, nous en donnons deux versions classiques à l'aide des notions suivantes:

\Def 
\sskip 
\begin{itemize}
    \item $f$ est quasi-concave si: $f(tx+ (1-t)y)\ge min(f(x),f(y)), \ t\in[0,1]$,
    \item $f$ est quasi-convexe si $-f$ est quasi-concave,
    \item $f$ est semi-continue supérieurement (scs) si, pour tout  $x_0$ et $\epsilon>0$, il existe un voisinage $U$ de $x_0$ tel que: $\forall x\in U, \ f(x)\le f(x_0)+\epsilon$,
    \item $f$ est sci si $-f$ est scs.
\end{itemize}
\bigskip
On constate par ailleurs que, pour $f$ scs sur un compact, alors $f$ atteint son maximum sur ce compact.
\bigskip 

\Theo 
\sskip 
Si $I$ et $J$ (ici sous-\ev) sont convexes, que l'un ou l'autre est compact, $g(\cdot,j)$ est quasi-concave et scs et $g(i,\cdot)$ est quasi-convexe et sci, alors $G$ a une valeur. De plus si $I$ (resp $J$) est compact, alors le joueur $1$ (resp $2$) a une stratégie optimale.
\end{document}
