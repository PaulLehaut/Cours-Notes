\documentclass{article}

\title{Statistiques et Analyse de Données}
\author{Paul Lehaut}
\usepackage{mesraccourcis}

\begin{document}
\maketitle
\newpage
\tableofcontents
\newpage
\section{Introduction}
La théorie des jeux est une branche des mathématiques qui étudie les interactions stratégiques. Elle s'applique dans de nombreux domaines comme l'économie, les sciences politiques, la biologie...
\sskip 
Le rôle de la théorie des jeux est à la fois de modéliser et d'analyser les situations. Idéalement, les compertements purement individuels permettraient d'atteindre des objectifs collectifs.
\bigskip 
\section{Jeux Sous Forme Normale: Stratégies Pures}
La forme normale est une manière standard de décrire formellement un jeu. On spécifie pour chaque joueur son ensemble de stratégies et sa fonction de paiement.
\sskip 
Dans ce chapitre (et le suivant), on supposera que les joueurs connaissent la description exacte du jeu.
\sskip 
On définit deux types de stratégies:
\begin{itemize}
    \item stratégie pure: spécifie, pour chaque situation, une décision bien déterminée
    \item stratégie mixte: elle intègre des choix aléatoires
\end{itemize}
dans ce chapitre, on se restreindra à des stratégies pures (désignées dès lors par stratégie).
\bigskip 
\subsection{Description d'un Jeu Sous Forme Normale}
Un jeu sous forme normale est défini pour un nombre fini $N$ de joueurs par:

\Def \ Jeu sous forme normale 
\sskip 
Un ensemble de stratégies $S_i$ pour chaque joueur $i$ et une fonction de paiement $g_i:\prod_{j=1}^N S_j\to\R$.
\sskip 
L'interprétation est la suivante: chaque joueur joue une fois, indépendamment des autres joueurs, et reçoit ensuite un paiement.
\bskip 
Il est intuitif de considérer que chaque joueur cherche à maximiser son profit. On étudie donc la question suivante: 

Si les joueurs sont rationnels, quelles stratégies vont-ils le plus probablement choisir ?
\bskip 
Dans ce chapitre, on se limite à des jeux statiques (les joueurs jouent une unique fois) et à information complète (les joueurs connaissent les paiements et stratégies des autres joueurs). On définit les classes de jeux particulièrement importantes suivantes:
\begin{itemize}
    \item Jeux à intérêts communs: tous les joueurs ont les mêmes intérêts ou préférences.
    \item Jeux à somme nulle: jeux à deux joueurs dont les intérêts sont antagonistes $ (g_1 = -g_2)$.
    \item Jeux de la bataille des sexes: ces jeux font intervenir une part de coordination et de conflit entre les agents.
    \item Jeux de la fureur de vivre: deux adolescents en voiture foncent l'un vers l'autre, aucun d'entre eux ne veut sortir de la route, si les deux sortes le résultat est neutre.
    \item Le dilemme du prisonnier: ce jeux fait ressortir une tension entre l'intérêt collectif et individuel, de nombreuses situations possèdent une structure similaire à celle-ci.
    \item Compétition en quantités dite de Cournot: deux entreprises produisent en quantité $q_i$ et à un coût $c_i(q_i)$, des biens identiques, le prix résultant de la loi de l'offre et de la demande est $p(q_1 + q_2)$, la situation se formalise alors par:
$$S_i = \R_+ \ \ \text{et} \ \ g_i(q_i, q_j) = q_i p(q_i + q_j) - c_i(q_i).$$
    \item Compétition en prix dite de Bertrand: il s'agit d'un modèle de compétition par les prix, chaque entreprise décide d'un prix de vente du bien et les consommateurs d'une quantité à acheter à la firme au prix le plus bas.
\end{itemize}
\bigskip 
\subsection{Stratégies Dominantes et Dominées}
Dans toute la suite on utilisera les notations suivantes: 
$$S = \prod_{j=1}^NS_j, \ \ S_{-i} = \prod_{j=1, \ j\ne i}^N S_j \ \ \text{et} g = (g_i).$$
on considèrera par ailleurs un jeu sous forme normale $\Gamma(N,S,g)$.
\bigskip 

\Def \ Stratégie strictement dominée 
\sskip 
Il s'agit d'une stratégie $s_i\in S_i$ telle qu'il existe $t_i\in S_i$ qui domine strictement $s_i$, c'est-à-dire:
$$\forall s_{-i}\in S_{-i}, \ g_i(t_i,s_{-i})>g_i(s_i,s_{-i}).$$
Un joueur 'rationnel' ne devrait jamais joué une stratégie strictement dominée. 
\bskip 
Une stratégie strictement dominante est une stratégie qui domine strictement toutes les autres, une telle stratégie, si elle existe, est évidemment unique.
\bigskip

\Def \ Equilibre en stratégies strictement dominantes
\sskip 
Il s'agit d'un profil de stratégies $s=(\ind s N)$ tel que, pour tout $i$, $s_i$ soit strictement dominante.
\bskip 
Si une telle stratégie existe, alors on peut considérer qu'il s'agit de la seule issue rationnelle du jeu.
On peut, par ailleurs, considérer une jeu $\Gamma(N,S^*,g)$ dans lequel on ne conserve que les stratégies strictement dominantes (si elles existent).
\end{document}
