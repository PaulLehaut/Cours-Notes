\documentclass{article}

\title{Analyse Complexe}
\author{Paul Lehaut}
\usepackage{mesraccourcis}

\begin{document}
\newcommand{\Cdiff}{$\mathbb{C}$-différentiable}
\newcommand{\Rdiff}{$\mathbb{R}$-différentiable}
\maketitle
\newpage
\tableofcontents
\newpage
\section{Différentiabilité Complexe}
\subsection{Préliminaires}
Dans tout ce cours, $\Omega$ désigne un ouvert de $\C$.
\bigskip

\Def \ Différentiabilité complexe 
\sskip 
Soit $f:A\subset \C\longrightarrow \C$, $f$ est \Cdiff \ en un point intérieur $a$ de $A$ si:
$$f'(a) = \lim_{h\to 0}\frac{f(a+h)-f(a)}{h} \ \ \text{existe dans } \C.$$
\sskip 
$f$ est \Cdiff \ sur $A$ si elle est dérivable en tout point de $A$, on dit alors qu'elle est holomorphe. Si $A=\C$, on dit que $f$ est complète.
\bigskip 

\Theo
\sskip 
$f$ holomorphe implique $f$ continue et $f'$ holomorphe.
\bskip 
\subsection{Dérivée et différentielle complexe}
On rappelle qu'un espace vectoriel est réel (resp complexe) si son champ de scalaire est l'axe des réels (resp le plan complexe).
\bigskip 

\Def 
\sskip 
Soient E et F des \ev \ normés complexes, soit $l:E\longrightarrow F$, $l$ est linéaire complexe si elle est additive est homogène complexe (ie $\forall \lambda\in \C l(\lambda z) = \lambda l(z)$).
\sskip 
Soientt $f:A\subset E\longrightarrow F$, $z$ un point intérieur de $A$, la différentielle complexe de $f$ en $z$ est l'opérateur linéaire complexe (donc continu):
$$df_z:E\longrightarrow F \ \ \text{tel que} \ \ \lim_{h\to 0}\frac{||f(z+h)-f(z)-df_z(h)||}{||h||}=0.$$
On a alors: $f(z+h)=f(z)+df_z(h)+\epsilon(h)||h||$ où $\epsilon(h)\to_{h\to0}0$.
\bigskip 

\Theo 
\sskip 
Soit $f:A\longrightarrow \C$ $df_z$ existe si et seulement si $f'(z)$ existe et alors: $df_z(h)=f'(z)h$.
\bigskip

\Prop 
\sskip 
On a les résultats classiques d'addition, de produit, de quotient et de composition des fonctions \Cdiff.
\bigskip 
\subsection{Equations de Cauchy-Riemann }
On note dans cette section $f=u+iv$, une fonction de $\Omega$ dans $\C$.
\bigskip 

\Theo
\sskip 
$f$ est \Cdiff \ si et seulement si on a l'une des équivalences suivantes:
\begin{itemize}
    \item $f$ est \Rdiff \ sur $\Omega$ et sa différentielle est $\C$-linéaire 
    \item $df_z(i)=idf_z(1)$ (1)
    \item $\frac{\partial f}{\partial x}=\frac{\partial f}{i\partial y}$ (2)
    \item $\frac{\partial u}{\partial x} = \frac{\partial v}{\partial y}$ et $\frac{\partial u}{\partial y} = -\frac{\partial v}{\partial x}$ (3)
\end{itemize}
Les égalités (2) et (3) correspondent aux équations de Cauchy-Riemann.
\bigskip 

\underline{Corrolaire :}
\sskip 
$f$ est \Cdiff \ si et seulement si $\frac{\partial f}{\partial x}$ et $\frac{\partial f}{\partial y}$ existent, sont continues et $\frac{\partial f}{\partial x}=\frac{\partial f}{i\partial y}$ ce qui équivaut également à $\frac{\partial u}{\partial x},\ \frac{\partial v}{\partial x}, \ \frac{\partial u}{\partial y}$ et $\frac{\partial v}{\partial y}$ existent, sont continues et vérifient l'égalité de Cauchy-Riemann définie précédemment.

\bigskip 
\section{Intégrale Linéaire et Primitive}
\subsection{Chemin}
L'objectif principal de cette session est d'obtenir la version complexe du théorème fondamentale du calcul intégral qui donne, dans le cas réel: 

Pour I un intervalle ouvert de $\R$, $a\in I$, $f:I\longrightarrow \R$ continue, alors $g$ est la primitive de $f$ si et seulement si:
$$\forall x\in I, \ g(x)=g(a)+\int_a^x f.$$
\bigskip 

\Def \ Chemin 
\sskip 
Un chemin $\gamma$ est une fonction continue de $[0,1]$ dans $\C$, si $A\subset \C$, $\gamma$ est un chemin de $A$ si son image est incluse dans $A$.
\sskip 
Un chemin est fermé si son point de départ est égal à son point d'arrivé ie: $\gamma(0)=\gamma(1)$. Des chemins sont consécutifs si le point de départ de l'un est le point final de l'autre.
\sskip 
Le chemin retour de $\gamma$ est le chemin $\gamma^*(t)=\gamma(1-t)$.
\bigskip 

\Def \ Ensembles ouverts (connexes par arcs)
\sskip 
$\Omega$ est connexe (par arcs) si pour chacun de ces points $x,y$ il y a un chemin de $\Omega$ qui lie $x$ à $y$.
\bigskip 

\Def \ Concaténation de chemins 
\sskip 
Soit $0=t_0<t_1<...<t_n=1$ une partition de $[0,1]$, la concaténation de chemins consécutifs associées à cette partition est le chemin $\gamma$ défini par:
$$\gamma = \gamma_1|_{t_1}...|_{t_{n-1}}\gamma_n \ \ \text{tel que } \forall k, \gamma|_{[t_{k-1},t_k]}=\gamma_k(\frac{t-t_{k-1}}{t_k-t_{k-1}}).$$
Si la partition est uniforme (ie $t_k=k/n$), alors on notera: $\gamma = \gamma_1|...|\gamma_n$.
\bigskip 

\Def \ Chemin rectifiable 
\sskip 
Un chemin $\gamma$ est rectifiable s'il est continûment différentiable par morceaux, on peut alors écrire $\gamma = \gamma_1|_{t_1}...|_{t_{n-1}}\gamma_n$ où les $\gamma_k$ sont continûments différentiables.
\bskip 
On peut donc introduire le lemme suivant pour renforcer la notion de connexité:
\smallskip 

\underline{Lemme:} Connexité et chemins rectifiables
\sskip
$\Omega$ est connexe si et seulement si toute paire de points de $\Omega$ peut être reliée par un chemin rectifiable dans $\Omega$.

\subsection{Intégrales de Ligne}
Pour étudier les intégrales de ligne on doit d'abord introduire la notion suivante:
\smallskip 

\Def \ Longueur d'un chemin rectifiable
\sskip 
La longueur d'un chemin $\gamma$ continûment différentiable est:
$$l(\gamma)=\int_0^1|\gamma'|\ge 0.$$
Si $\gamma$ est seulement un chemin rectifiable, alors:
$$l(\gamma)=\sum_{k=1}^nl(\gamma_k).$$
On peut alors définir:
\smallskip 

\Def \ Intégrale de ligne 
\sskip 
L'intégrale de ligne le long d'un chemin continûment différentiable $\gamma$ d'une fonction complexe $f$ définie et continue sur l'image de $\gamma$ est le nombre complexe défini par:
$$\int_\gamma f = \int_0^1f(\gamma(t))\gamma'(t)dt.$$
Si $\gamma$ est seulement un chemin rectifiable, alors:
$$\int_\gamma f =\sum_{k=1}^n\int_{\gamma_k}f.$$
\bigskip 

\Prop
\sskip 
L'intégrale de ligne est $\C$-linéaire.
\sskip 
Pour $\gamma^*$ le chemin retour de $\gamma$, alors $l(\gamma)=l(\gamma^*)$ et $\int_\gamma f = -\int_{\gamma^*}f$.
\bigskip 

\Theo \ Inégalité M-L 
\sskip 
Pour tout chemin rectifiable $\gamma$ et toute fonction continue $f:\gamma([0,1])\longrightarrow\C$:
$$|\int_{\gamma}f|\le(\max_{z\gamma([0,1])}|f(z)|)l(\gamma).$$
\bigskip 

\underline{Corrolaire:} Convergence dans les intégrales de ligne 
\sskip 
Pour tout chemin rectifiable $\gamma$ et toute suite de fonction continue $f_n:\gamma([0,1])\longrightarrow\C$ qui converge uniformément vers $f$, il vient:
$$\lim_{n\to\pinf}\int_\gamma f_n = \int_\gamma f.$$
\bigskip 

\Theo \ Invariance par re-paramétrisation 
\sskip 
Soit $\gamma$ un chemin continûment différentiable, soit $\Phi:[0,1]\longrightarrow[0,1]$ un difféomorphisme de classe $\mc C^1$ strictement croissant tel que $\Phi(0)=0$ et $\Phi(1)=1$, alors il vient:
\begin{itemize}
    \item Le chemin $\mu = \gamma\circ\Phi$ est continûment différentiable, son image est la même que celle de $\gamma$,
    \item Les longueurs de $\mu$ et de $\gamma$ sont identiques,
    \item Pour toutes fonctions continues $f:\gamma([0,1])\longrightarrow\C$:
    $$\int_\gamma f = \int_\mu f.$$
\end{itemize}
\bigskip 

\Def \ Image d'un  chemin par un fonction 
\sskip 
Soit un chemin $\gamma$ et une fonctions continues $f:\gamma([0,1])\longrightarrow\C$, l'image de $\gamma$ par $f$ est le chemin $f\circ\gamma$.
\bigskip 

\Theo \ Changement de variable 
\sskip 
Soit $\gamma$ un chemin rectifiable sur $\Omega$ et $f$ une fonction \Cdiff, alors le chemin $f\circ \gamma$ est rectifiable et pour toute fonction continue $g:f\circ \gamma([0,1])\longrightarrow\C$:
$$\int_{f\circ \gamma}g = \int_\gamma (g\circ f)(t) f'(t)dt.$$
\bigskip 
\subsection{Primitives}
On va définir la primitive de façon analogue à la primitive de l'analyse réelle:
\smallskip

\Def \ Primitive
\sskip
Soit $f:\Omega\longrightarrow \C$, une primitive (ou antidérivée) de $f$ est une fonction holomorphique $g:\Omega\longrightarrow \C$ telle que $g'=f$.
\bigskip 

\Theo \ Théorème fondamental du calcul intégral 
\sskip 
Soit $f:\Omega\longrightarrow \C$ une fonction continue et soit $a$ un point de de $\Omega$. Une fonction $g:\Omega\longrightarrow \C$ est une primitive de $f$ si et seulement si pour tout $z\in\Omega$ et tout chemin rectifiable $\gamma$ de $\Omega$ qui lient $a$ et $z$, alors:
$$g(z)=g(a)+\int_\gamma f(w)dw.$$
\bigskip 

\underline{Corrolaire:} Existence de primitives et ensemble de primitives
\sskip 
On suppose que $\Omega$ est connexe, la fonction $f:\Omega\longrightarrow\C$ admet une primitive si et seulement si elle est continue et si pour tout chemin fermé rectifiable $\gamma$:
$$\int_\gamma f =0.$$
\sskip 
Maintenant, si $g:\Omega\longrightarrow \C$ est une primitive de $f$, alors la fonction $h:\Omega\longrightarrow\C$ est également une primitive de $f$ si et seulement si elle est égale à $g$ à une constant près.
\bigskip 

\underline{Corrolaire:} Intégration par parties 
\sskip 
On suppose que $\Omega$ est connexe, soit $\gamma$ un chemin rectifiable de $\Omega$, pour toute paire de fonction holomorphe $f,g:\Omega\longrightarrow\C$, alors:
$$\int_\gamma f'g = [fg(\gamma(1))-fg(\gamma(0))]-\int_\gamma fg'.$$
\end{document}