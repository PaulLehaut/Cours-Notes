\documentclass{article}

\title{Analyse Complexe}
\author{Paul Lehaut}
\usepackage{mesraccourcis}

\begin{document}
\newcommand{\Cdiff}{$\mathbb{C}$-différentiable}
\newcommand{\Rdiff}{$\mathbb{R}$-différentiable}
\maketitle
\newpage
\tableofcontents
\newpage
\section{Différentiabilité Complexe}
\subsection{Préliminaires}
Dans tout ce cours, $\Omega$ désigne un ouvert de $\C$.
\bigskip

\Def \ Différentiabilité complexe 
\sskip 
Soit $f:A\subset \C\longrightarrow \C$, $f$ est \Cdiff \ en un point intérieur $a$ de $A$ si:
$$f'(a) = \lim_{h\to 0}\frac{f(a+h)-f(a)}{h} \ \ \text{existe dans } \C.$$
\sskip 
$f$ est \Cdiff \ sur $A$ si elle est dérivable en tout point de $A$, on dit alors qu'elle est holomorphe. Si $A=\C$, on dit que $f$ est complète.
\bigskip 

\Theo
\sskip 
$f$ holomorphe implique $f$ continue et $f'$ holomorphe.
\bskip 
\subsection{Dérivée et différentielle complexe}
On rappelle qu'un espace vectoriel est réel (resp complexe) si son champ de scalaire est l'axe des réels (resp le plan complexe).
\bigskip 

\Def 
\sskip 
Soient E et F des \ev \ normés complexes, soit $l:E\longrightarrow F$, $l$ est linéaire complexe si elle est additive est homogène complexe (ie $\forall \lambda\in \C l(\lambda z) = \lambda l(z)$).
\sskip 
Soientt $f:A\subset E\longrightarrow F$, $z$ un point intérieur de $A$, la différentielle complexe de $f$ en $z$ est l'opérateur linéaire complexe (donc continu):
$$df_z:E\longrightarrow F \ \ \text{tel que} \ \ \lim_{h\to 0}\frac{||f(z+h)-f(z)-df_z(h)||}{||h||}=0.$$
On a alors: $f(z+h)=f(z)+df_z(h)+\epsilon(h)||h||$ où $\epsilon(h)\to_{h\to0}0$.
\bigskip 

\Theo 
\sskip 
Soit $f:A\longrightarrow \C$ $df_z$ existe si et seulement si $f'(z)$ existe et alors: $df_z(h)=f'(z)h$.
\bigskip

\Prop 
\sskip 
On a les résultats classiques d'addition, de produit, de quotient et de composition des fonctions \Cdiff.
\bigskip 
\subsection{Equations de Cauchy-Riemann }
On note dans cette section $f=u+iv$, une fonction de $\Omega$ dans $\C$.
\bigskip 

\Theo
\sskip 
$f$ est \Cdiff \ si et seulement si on a l'une des équivalences suivantes:
\begin{itemize}
    \item $f$ est \Rdiff \ sur $\Omega$ et sa différentielle est $\C$-linéaire 
    \item $df_z(i)=idf_z(1)$ (1)
    \item $\frac{\partial f}{\partial x}=\frac{\partial f}{i\partial y}$ (2)
    \item $\frac{\partial u}{\partial x} = \frac{\partial v}{\partial y}$ et $\frac{\partial u}{\partial y} = -\frac{\partial v}{\partial x}$ (3)
\end{itemize}
Les égalités (2) et (3) correspondent aux équations de Cauchy-Riemann.
\bigskip 

\underline{Corrolaire :}
\sskip 
$f$ est \Cdiff \ si et seulement si $\frac{\partial f}{\partial x}$ et $\frac{\partial f}{\partial y}$ existent, sont continues et $\frac{\partial f}{\partial x}=\frac{\partial f}{i\partial y}$ ce qui équivaut également à $\frac{\partial u}{\partial x},\ \frac{\partial v}{\partial x}, \ \frac{\partial u}{\partial y}$ et $\frac{\partial v}{\partial y}$ existent, sont continues et vérifient l'égalité de Cauchy-Riemann définie précédemment.

\bigskip 
\section{Intégrale Linéaire et Primitive}
\subsection{Chemin}
L'objectif principal de cette session est d'obtenir la version complexe du théorème fondamentale du calcul intégral qui donne, dans le cas réel: 

Pour I un intervalle ouvert de $\R$, $a\in I$, $f:I\longrightarrow \R$ continue, alors $g$ est la primitive de $f$ si et seulement si:
$$\forall x\in I, \ g(x)=g(a)+\int_a^x f.$$
\bigskip 

\Def \ Chemin 
\sskip 
Un chemin $\gamma$ est une fonction continue de $[0,1]$ dans $\C$, si $A\subset \C$, $\gamma$ est un chemin de $A$ si son image est incluse dans $A$.
\sskip 
Un chemin est fermé si son point de départ est égal à son point d'arrivé ie: $\gamma(0)=\gamma(1)$. Des chemins sont consécutifs si le point de départ de l'un est le point final de l'autre.
\sskip 
Le chemin retour de $\gamma$ est le chemin $\gamma^*(t)=\gamma(1-t)$.
\bigskip 

\Def \ Ensembles ouverts (connexes par arcs)
\sskip 
$\Omega$ est connexe (par arcs) si pour chacun de ces points $x,y$ il y a un chemin de $\Omega$ qui lie $x$ à $y$.
\bigskip 

\Def \ Concaténation de chemins 
\sskip 
Soit $0=t_0<t_1<...<t_n=1$ une partition de $[0,1]$, la concaténation de chemins consécutifs associées à cette partition est le chemin $\gamma$ défini par:
$$\gamma = \gamma_1|_{t_1}...|_{t_{n-1}}\gamma_n \ \ \text{tel que } \forall k, \gamma|_{[t_{k-1},t_k]}=\gamma_k(\frac{t-t_{k-1}}{t_k-t_{k-1}}).$$
Si la partition est uniforme (ie $t_k=k/n$), alors on notera: $\gamma = \gamma_1|...|\gamma_n$.
\bigskip 

\Def \ Chemin rectifiable 
\sskip 
Un chemin $\gamma$ est rectifiable s'il est continûment différentiable par morceaux, on peut alors écrire $\gamma = \gamma_1|_{t_1}...|_{t_{n-1}}\gamma_n$ où les $\gamma_k$ sont continûments différentiables.
\bskip 
On peut donc introduire le lemme suivant pour renforcer la notion de connexité:
\smallskip 

\underline{Lemme:} Connexité et chemins rectifiables
\sskip
$\Omega$ est connexe si et seulement si toute paire de points de $\Omega$ peut être reliée par un chemin rectifiable dans $\Omega$.

\subsection{Intégrales de Ligne}
Pour étudier les intégrales de ligne on doit d'abord introduire la notion suivante:
\smallskip 

\Def \ Longueur d'un chemin rectifiable
\sskip 
La longueur d'un chemin $\gamma$ continûment différentiable est:
$$l(\gamma)=\int_0^1|\gamma'|\ge 0.$$
Si $\gamma$ est seulement un chemin rectifiable, alors:
$$l(\gamma)=\sum_{k=1}^nl(\gamma_k).$$
On peut alors définir:
\smallskip 

\Def \ Intégrale de ligne 
\sskip 
L'intégrale de ligne le long d'un chemin continûment différentiable $\gamma$ d'une fonction complexe $f$ définie et continue sur l'image de $\gamma$ est le nombre complexe défini par:
$$\int_\gamma f = \int_0^1f(\gamma(t))\gamma'(t)dt.$$
Si $\gamma$ est seulement un chemin rectifiable, alors:
$$\int_\gamma f =\sum_{k=1}^n\int_{\gamma_k}f.$$
\bigskip 

\Prop
\sskip 
L'intégrale de ligne est $\C$-linéaire.
\sskip 
Pour $\gamma^*$ le chemin retour de $\gamma$, alors $l(\gamma)=l(\gamma^*)$ et $\int_\gamma f = -\int_{\gamma^*}f$.
\bigskip 

\Theo \ Inégalité M-L 
\sskip 
Pour tout chemin rectifiable $\gamma$ et toute fonction continue $f:\gamma([0,1])\longrightarrow\C$:
$$|\int_{\gamma}f|\le(\max_{z\gamma([0,1])}|f(z)|)l(\gamma).$$
\bigskip 

\underline{Corrolaire:} Convergence dans les intégrales de ligne 
\sskip 
Pour tout chemin rectifiable $\gamma$ et toute suite de fonction continue $f_n:\gamma([0,1])\longrightarrow\C$ qui converge uniformément vers $f$, il vient:
$$\lim_{n\to\pinf}\int_\gamma f_n = \int_\gamma f.$$
\bigskip 

\Theo \ Invariance par re-paramétrisation 
\sskip 
Soit $\gamma$ un chemin continûment différentiable, soit $\Phi:[0,1]\longrightarrow[0,1]$ un difféomorphisme de classe $\mc C^1$ strictement croissant tel que $\Phi(0)=0$ et $\Phi(1)=1$, alors il vient:
\begin{itemize}
    \item Le chemin $\mu = \gamma\circ\Phi$ est continûment différentiable, son image est la même que celle de $\gamma$,
    \item Les longueurs de $\mu$ et de $\gamma$ sont identiques,
    \item Pour toutes fonctions continues $f:\gamma([0,1])\longrightarrow\C$:
    $$\int_\gamma f = \int_\mu f.$$
\end{itemize}
\bigskip 

\Def \ Image d'un  chemin par un fonction 
\sskip 
Soit un chemin $\gamma$ et une fonctions continues $f:\gamma([0,1])\longrightarrow\C$, l'image de $\gamma$ par $f$ est le chemin $f\circ\gamma$.
\bigskip 

\Theo \ Changement de variable 
\sskip 
Soit $\gamma$ un chemin rectifiable sur $\Omega$ et $f$ une fonction \Cdiff, alors le chemin $f\circ \gamma$ est rectifiable et pour toute fonction continue $g:f\circ \gamma([0,1])\longrightarrow\C$:
$$\int_{f\circ \gamma}g = \int_\gamma (g\circ f)(t) f'(t)dt.$$
\bigskip 
\subsection{Primitives}
On va définir la primitive de façon analogue à la primitive de l'analyse réelle:
\smallskip

\Def \ Primitive
\sskip
Soit $f:\Omega\longrightarrow \C$, une primitive (ou antidérivée) de $f$ est une fonction holomorphique $g:\Omega\longrightarrow \C$ telle que $g'=f$.
\bigskip 

\Theo \ Théorème fondamental du calcul intégral 
\sskip 
Soit $f:\Omega\longrightarrow \C$ une fonction continue et soit $a$ un point de de $\Omega$. Une fonction $g:\Omega\longrightarrow \C$ est une primitive de $f$ si et seulement si pour tout $z\in\Omega$ et tout chemin rectifiable $\gamma$ de $\Omega$ qui lient $a$ et $z$, alors:
$$g(z)=g(a)+\int_\gamma f(w)dw.$$
\bigskip 

\underline{Corrolaire:} Existence de primitives et ensemble de primitives
\sskip 
On suppose que $\Omega$ est connexe, la fonction $f:\Omega\longrightarrow\C$ admet une primitive si et seulement si elle est continue et si pour tout chemin fermé rectifiable $\gamma$:
$$\int_\gamma f =0.$$
\sskip 
Maintenant, si $g:\Omega\longrightarrow \C$ est une primitive de $f$, alors la fonction $h:\Omega\longrightarrow\C$ est également une primitive de $f$ si et seulement si elle est égale à $g$ à une constant près.
\bigskip 

\underline{Corrolaire:} Intégration par parties 
\sskip 
On suppose que $\Omega$ est connexe, soit $\gamma$ un chemin rectifiable de $\Omega$, pour toute paire de fonction holomorphe $f,g:\Omega\longrightarrow\C$, alors:
$$\int_\gamma f'g = [fg(\gamma(1))-fg(\gamma(0))]-\int_\gamma fg'.$$
\bigskip 
\section{Espaces Connexes}
\subsection{Espaces Connexes (par Arcs)}
Ils existent deux définitions légèrements différentes qui caractérise le fait d'être en un seul morceau: la connexité et la connexité par arcs. La connexité est un peu plus faible que la connexité par arcs et donc un peu plus robuste et puissante. Néanmoins, dans la majorité des cas, et pour tous les ouverts, ces deux propriétés sont équivalentes.
\smallskip 

\Def \ Espace connexe par arcs 
\sskip 
Un ensemble $A$ est connexe par arcs si chacune de ces paires de points peut être reliée par un chemin de $A$.
\bigskip 

\Def \ Dilatation
\sskip 
$B$ est une dilatation de $A$ si on peut écrire : $B=\cup_{a\in A} B(a,r_a)$ avec $r_a>0$ pour tout $a$.
\bigskip 

\Def \ Espace connexe 
\sskip 
Un espace est connexe si toute ses dilatations sont connexes par arcs.
\bigskip 

\Theo 
\sskip
Tout ensemble connexe par arcs est connexe, de façon réciproque, tout ensemble ouvert connexe est connexe par arcs. 
\smallskip 
Il est donc immédiat qu'un ensemble ouvert est connexe si et seulement s'il est connexe par arcs.
\bigskip 
\subsection{Opérations Ensemblistes}
De nombreuses propriétés sont communes aux espaces connexes et connexes par arcs, c'est le cas, sauf mention contraire de toutes les propriétés suivantes:
\bigskip 

\Theo \ Union d'ensemble d'intersection non-vide 
\sskip 
Si $\mc A$ est une collection d'ensemble connexe (par arcs) dont l'intersection est non vide, alors leur union est également connexe (par arcs). 
\bskip 
On peut par ailleurs se convaincre facilement que l'union de deux ouverts d'intersection vide n'est pas connexe (par arcs).
\bigskip

\Theo \ Fermeture d'espaces connexes 
La fermeture d'un espace connexe est connexe. 
\sskip 
Attention, ce théorème ne s'applique en général pas aux espaces connexes par arcs, considérer par exemple l'ensemble connexe par arcs $A=\{(x,\sin(1/x)); \ x\in ]0,1]\}$ dont la fermeture est $A\cup \{(0,y); \ y\in [-1,1]\}$.
\bigskip 
\subsection{Composantes}
On introduit deux concepts de composantes selon le type de connexité:
\smallskip 

\Def \ Composante 
\sskip 
Une composante (connexe (par arcs)) d'un ensemble non vide $A$ est un sous-ensemble de $A$ qui est connexe (par arcs) et maximal au sens de l'inclusion pour de tels ensembles.
\bigskip 

\Theo \ Partition en composantes 
\sskip 
Les composantes d'un ensemble non vide forment une partition de $A$. 
\bskip 
On en déduit immédiatement qu'un ensemble non vide est connexe (par arcs) si et seulement s'il possède une unique composante. 
\bigskip 

\Theo \ Composantes d'un ensemble ouvert 
\sskip 
Les partitions en composante connexe et en composante connex par arcs d'un ensemble ouvert non vide sont indentiques.
\bskip 
On peut finalement introduire le concept de fonction localement constante:
\smallskip 

\Def \ Fonction localement constante 
\sskip 
Un fonction $f$ sur $A$ est localement constante si, pour tout point de $A$, il existe une boule ouverte non vide centrée sur ce point telle que $f$ soit constante sur l'intersection de $A$ et de cette boule. 
\bigskip 

\Theo \ Fonction localement constante et ensemble connexe 
\sskip 
Un ensemble est connexe si et seulement si toute fonction localement constante définie sur cet ensemble est en fait constante.
\bskip 

\section{Théorème Intégral de Cauchy - Version Locale}
L'objectif de ce chapitre est d'obtenir une première version du théorème intégral de Cauchy:
\smallskip 

\Theo \ Intégrale de Cauchy (Version locale)
\sskip 
Soit $f:\Omega\longrightarrow \C$, \Cdiff, pour tout $a\in\Omega$, il existe un rayon $r>0$ tel que $B(a,r)\subset \Omega$ et pour tout chemin rectifiable fermé $\gamma$ de cette boule:
$$\int_\gamma f = 0.$$
\bskip 
\subsection{Lemme d'Intégration sur les Chemins Polygonaux}
On considère dans ce chapitre une fonction holomorphe $f:\Omega \longrightarrow \C$.
\bigskip 

\underline{Lemme:} Intégration de triangles 
\sskip
Si $\Delta\subset\Omega$ est un triangle dont les sommets sont $a,b$ et $c$, ie $\Delta = \{\lambda a + \mu b + \nu c\ ;\ \lambda,\mu,\nu\ge0 \ \ \text{et } \lambda+\mu+\nu = 1\}$, et si $\gamma = [a\to b\to c\to a]$ est un chemin 'en ligne droite' qui forme la frontière de $\Delta$, alors:
$$\int_\gamma f =0.$$
\bigskip 

\Def \ Ensemble en étoile 
\sskip 
Un ensemble $A$ est étoilé s'il existe $a\in A$ tel que pour tout point $z\in A$, le segment $[a,z]$ soit inclus dans $A$. 
\bigskip 

\underline{Lemme:} Intégration sur les chemins polygonaux
\sskip 
Supposons que $\Omega$ soit étoilé, pour tout chemin polygonal fermé $\gamma = [a_0\to ...\to a_{n-1}\to a_0]$ de $\Omega$, alors: 
$$\int_\gamma f =0.$$
\bskip 

\subsection{Approximation de Chemin Rectifiable par des Polygones}
L'objectif de cette partie est d'étendre le lemme précédent à des chemins plus généraux que les chemins polygonaux fermés.
\smallskip 

\underline{Lemme:} Approximation polynomiale de chemin rectifiable 
\sskip 
Soit $\gamma$ un chemin rectifiable, pour tout $\epsilon_l>0$ et $\epsilon_\infty>0$, il y a un chemin polygonale orienté $\mu$ avec les mêmes points de départs et d'arrivés que $\gamma$ tel que:
$$l(\mu-\gamma)\le\epsilon_l \ \ \text{et} \ \ |(\mu - \gamma)(t)|\le \epsilon_\infty.$$
\bskip
\subsection{Théorème Intégral de Cauchy}
On peut alors énoncer un premier théorème:
\smallskip 

\Theo \ Intégrale de Cauchy (Version étoilée)
\sskip 
Soit $f:\Omega\longrightarrow \C$, \Cdiff \ où $\Omega$ est supposé étoilé, pour tout chemin rectifiable fermé $\gamma$, l'intégral de chemin de $f$ selon $\gamma$ est nulle.
\bskip 
On en déduit le théorème suivant: 
\smallskip 

\Theo \ Intégrale de Cauchy pour les disques 
\sskip 
On suppose désormais que $\gamma$ forme un disque de rayon $r$ et de centre $c$ avec $B_F(c,r)\subset\Omega$, $\Omega$ est simplement un ouvert de $\C$, alors pour toute fonction holomorphe $f:\Omega\longrightarrow \C$:
$$\forall z\in B(c,r), \ f(z)=\frac{1}{2i\pi}\int_\gamma \frac{f(w)}{w-z}dw.$$
\bskip 
Un corrolaire immédiat de ce résultat est que la dérivée d'une fonction holomorphe est encore holomorphe.
\bigskip 

\Theo \ Morera 
\sskip 
Soit $f:\Omega\longrightarrow\C$, $f$ est holomorphe si et seulement si elle est continue et, localement, son intégrale de ligne selon des chemins rectifiables fermés est nulle ie:
$$\forall c\in\Omega,\ \exists r>0: \ B(c,r)\subset\Omega\ \ \text{et } \forall \gamma \ \ \text{chemin fermé rectifiable de }B(c,r), \ \int_\gamma f =0.$$
\bigskip 

\Theo \ Limite de fonction holomorphique 
\sskip 
Si une suite $(f_n)$ de fonction holomorphique de $\Omega$ dans $\C$ converge localement uniformément vers $f$, ie:
$$\forall c\in \Omega, \ \exists r>0: \ B(c,r)\subset \Omega \ \ \text{et } \lim_{n\to\infty}||f_n(z) - f(z)||_\infty = 0$$
alors $f$ est holomorphe.
\bigskip 

\Theo \ Liouville 
\sskip 
Toute fonction holomorphe sur $\C$ bornée est constante.
\bskip 
\section{Fonctions Analytiques}
\subsection{Convergence de Séries de Fonctions}
Cette section rappelle des résultats classiques sur les séries de fonctions.
\smallskip 

\Def \ Série entière 
\sskip 
Soient $c\in \C$ et $(a_n)\in\C^\N$, on peut définir la série entière: 
$$\sum_n a_n (z-c)^n$$
le rayon de convergence $R$ de cette série définit le disque $B(c,R)$ sur lequel cette série est absolument convergente.
\smallskip 

\Theo \ Calcul du rayon de convergence 
\sskip 
On définit le rayon de convergence comme: $r=\frac{1}{\lim\sup_n|a_n|^{1/n}}$.
\bskip 
On peut également calculer $R$ à l'aide de la méthode suivante:

On s'intéresse à la borne de croissance de la suite $(a_n)$ définie par: $\sigma_0 = \inf\{\sigma\in[0,\pinf] | \ |a_n|\le\sigma^n \ \text{APCR}\}$, alors $R=1/\sigma_0$.
\bigskip 

\Theo \ Dérivation terme à terme 
\sskip 
On s'intéresse à $f$ définie sur $B(c,R)$ par $f(z)=\sum_{n=0}^{\pinf }a_n (z-c)^n$, alors cette fonction est continue et dérivable et sa dérivée est:
$$f'(z)=\sum_{n=0}^{\pinf} a_{n+1}(n+1)(z-c)^n$$
définie avec le même rayon de convergence $R$.
\bigskip 

\Theo \ Intégration terme à terme 
\sskip 
La somme $\sum_{n} a_n (z-c)^n$ définie sur $B(c,R)$ est intégrable et son intégrale se calcule terme à terme: 
$$\int_\gamma\sum_{n=0}^{\pinf} a_n (w-c)^ndw = \sum_{n=0}^{\pinf}\int_\gamma a_n(w-c)^ndw.$$
\bskip 
\subsection{Fonction Holomorphique et Série Entière}
On commence par rappeler le résultat suivant:
\smallskip 

\Def \ Série de Taylor
\sskip 
La série de Taylor d'une fonction $f$ est, sous réserve d'existence, définie par: 
$$f(z)=\sum_{n=0}^{\pinf}\frac{f^{(n)}(c)}{n!}(z-c)^n$$
\smallskip 

\Theo 
\sskip
Si une fonction $f:\Omega\longrightarrow \C$ est développable en série entière sur $B(c,R)\subset\Omega$, alors cette série est sa série de Taylor.
\bskip 
Une fonction développable en série entière au voisinage de chacun des points de son domaine de définition est appelée une fonction analytique.
\bigskip 

\Theo \ Fonction analytique et holomorphe
\sskip 
Une fonction est analytique si et seulement si elle est holomorphe.
\bskip 
\subsection{Séries de Laurent}
On appelle anneau de centre $c$, de rayon intérieur $r_1$ et extérieur $r_2$ l'ensemble: 
$$A(c,r_1,r_2)=\{z\in\C|\ r_1<|z-c|<r_2 \}.$$

\Def \ Série de Laurent
\sskip 
La série de Laurent centrée en $c$ est définie par:
$$\sum_{n=\ninf}^{\pinf}a_n(z-c)^n.$$
Cette série est convergente pour $z\in\C$ si les séries $\sum_n a_n(z-c)^n$ et $\sum_{n\in\N^*} a_{-n}(z-c)^{-n}$ le sont. Dans le cas de convergence, sa somme est définie comme la somme des deux séries précédentes.
\bigskip 

\Theo \ Convergence des séries de Laurent 
\sskip 
Les rayons de convergence intérieur $r_1$ et extérieur $r_2$ définis par:
$$r_1=\limsup_n |a_{-n}|^{1/n}\ \ \text{et} \ \ r_2=\frac{1}{\limsup_n |a_{-n}|^{1/n}}$$
sont tels que la série de Laurent $\sum_{n\in\Z}a_n(z-c)^n$ converge uniquement dans $A(c,r_1,r_2)$, dans cette anneaux, la convergence est localement normale.
\bigskip 

\Theo \ Série de Laurent et fonction holomorphique
\sskip 
Pour toute fonction holomorphe $f$ définie sur $\Omega$, pour tout anneau $A(c,r_1,r_2)\subset\Omega$, $f$ est développable en série de Laurent sur cet anneau.
\bskip 
\section{Zeros et Pôles}
On introduit le concept de point distant, un point $a\in Z\subset\C$ est distant s'il est au moins à une distance $d$ des autres points de $Z$.
\bskip 
\subsection{Zeros}
On considère $f:\Omega\longrightarrow\C$ holomorphique.
\smallskip 

\Def 
\sskip 
Un point $a\in\Omega$ est un zéro de $f$ si $f(a)=0$.Le point $a$ est de multiplicité $p$ si:
$$\exists a^*\in\C^*:\ f(z)\sim_{z\to a} a^*(z-a)^p.$$

\Theo 
\sskip 
Un zéro de $f$ est de multiplicité $p$ si et seulement si:
$$f'(a)=f^{(2)}(a)=...=f^{(p-1)}=0 \ \ \text{et } f^{(p)}\neq 0.$$
\bskip 
On peut alors écrire la série de Taylor de $f$ en $a$:
$$f(z)=\sum_{n=p}^{\pinf}a_n(z-a)^n$$
avec $a_p\neq 0$.
\bigskip 

\Theo \
Si $a$ est un zéro d'ordre $p$ de $f$, alors il existe $h:\Omega\longrightarrow\C$ \Cdiff \ telle que:
$$\forall z\in\Omega, \ f(z)=h(z)(z-a)^p \ \ \text{et} \ \ h(a)\neq 0.$$
\bigskip 

\underline{Lemme:}
\sskip 
Si $a$ est un zéro de $f$ de multiplicité infinie, alors $f$ est nulle sur $\Omega$ si $\Omega$ est connexe.
\bigskip 

\underline{Lemme:}
Soit $a$ un zéro de $f$ de multiplicité $p$ finie, alors $a$ est isolé ie:
$$\exists r\in \R_+^*:\ \forall z\in B(a,r), \ f(z)=0 \implies  z=a. $$
\bigskip 

\Theo \ Zéros isolés 
\sskip 
Si $f$ n'est pas constante nulle, alors ces zéros sont isolés.
\bskip 
\subsection{Pôles et singularités}
Soit $c$ un singularité distante de $f$.
\smallskip 

\Def \ Singularité supprimable 
$c$ est supprimable s'il existe $h:\Omega\cup\{c\}\longrightarrow \C$ \Cdiff \ telle que $h(z)=f(z)$ pour tout $z\in\Omega$.
\bigskip 

\Def \ Pôle de multiplicité $p\in \N^*$
\sskip 
Il s'agit d'une singularité telle qu'il existe $a^*\in\C^*$ tel que $f(z)\sim_{z\to c}\frac{a^*}{(z-c)^p}$.
\bskip 
Les autres singularités sont dites essentielles.
\bigskip 

\Theo \ Caractérisation des singularités supprimables 
\sskip 
Une singularité distante $c$ est supprimable si et seulement si:
\begin{itemize}
    \item $f(z)\longrightarrow _{z\to c} a^*\in \C$
    \item $f$ est bornée sur $A(c,0,r)$
    \item La série de Laurent de $f$ en $c$ est une série entière.
\end{itemize}
\bigskip 

\Theo \ Caractérisation de la multiplicité d'un pôle 
\sskip 
Une singularité distante $c$ est un pôle de multiplicié $p$ si et seulement si:
\begin{itemize}
    \item La série de Laurent de $f$ sur $A(c,0,r)$ est $f(z)=\sum_{-p}^{\pinf}a_nz^n$
    \item il existe $h:\Omega\longrightarrow\C$ holomorphe telle que $h(c)\neq0$ et $f(z)=\frac{h(c)}{(z-c)^p}$ pour tout $z\in\Omega$
\end{itemize}
\bskip 
\subsection{Calcul des Résidus}
On considère toujours $c$ une singularité isolée de $f$.
\smallskip 

\Theo 
\sskip 
Si la série de Laurent de $f$ sur $A(c,0,r)$ est: 
$$f(z)=\sum_{n=\ninf}^{\pinf}a_n(z-c)^n$$
alors $res(f,c)=a_{-1}$.
\bigskip 

\underline{Corrolaires}
\sskip 
\begin{itemize}
    \item Si $c$ est un pôle de multiplicité $p$, alors: 
$$res(f,c)=\lim_{z\to c}\frac{1}{(p-1)!}\frac{d^{p-1}(f(z)(z-c)^p)}{dz^{p-1}}$$
    \item $c$ est un pôle simple de $f$ si et seulement si $f(z)(z-c)\longrightarrow_{z\to c} a$, alors: $res(f,c)=\lim_{z\to c}f(z)(z-c)$
    \item pour $g,h:\Omega\longrightarrow \C$ \Cdiff, si $f=\frac{g}{h}$, supposons que $g(c)\neq 0$, $h(c)=0$ et $h'(c)=0$, alors $res(f,c)=\frac{g(c)}{h'(c)}$
\end{document}